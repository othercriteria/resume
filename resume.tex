\documentclass{article}
% Starting with the following template:
% 
% Resume template for Brown's joint CareerLAB and Science Center
% LaTeX/Resume workshop
% (http://bit.ly/resumeTemplate)

% Shrink margins
\usepackage[margin=0.7in]{geometry}

% For hyperlinks
\usepackage{hyperref}

% For making compact lists with itemize* environment
\usepackage{mdwlist}

% Fancy sections
\newcommand{\ressection}[1]{\noindent{\large\textbf{#1}}\vspace{2pt}\hrule\vspace{4pt}}

% For items with dates: \hfill fills up horizontal space until the
% words to the left and the right are as far apart as possible
\newcommand{\leftandright}[2]{\noindent\textbf{#1}\hfill\textbf{#2}}

% Remove page numbering
\pagestyle{empty}

\begin{document}

%% Change font family
\sffamily

\begin{center}
\textbf{\huge{Daniel~L.~Klein}}

web: \href{http://mesokurtosis.com}{mesokurtosis.com}
\textbullet\, github: \href{https://gititbub.com/othercriteria}{othercriteria}
\textbullet\, email: \href{mailto:daniel_klein@brown.edu}{daniel\_klein@brown.edu}
\textbullet\, cell: 952.454.4197
\end{center}

\ressection{Education}

\leftandright{Brown University, \textmd{Ph.D. Applied
    Mathematics}}{Providence, RI \textbullet\ May, 2015 (anticipated)}
\begin{itemize*}
\item Advised by \href{http://www.dam.brown.edu/people/harrison}{Matt
    Harrison}, with research focus on statistical inference in
  settings of extreme data sparsity or imbalance.
\item TA for Intro Stats, Math Stats I/II, Recent Applications in
  Probability and Statistics.
\item Organized and hosted speaker visits and meetings for Pattern
  Theory seminar.
\end{itemize*}

\leftandright{Williams College, \textmd{B.A.\ Mathematics and
    Biology}}{Williamstown, MA \textbullet\, June, 2006}
\begin{itemize*}
\item Departmental honors in Biology, with thesis ``Understanding
  aggregation in the membracid \emph{Publilia concava}: using models
  to disentangle processes''.
\end{itemize*}

\vspace{1.0em}

\ressection{Work experience}

\leftandright{TransForm Pharmaceuticals, Inc.}{Lexington, MA
  \textbullet\, September 2006 -- August 2008} \\
\textit{Assistant Scientist, Scientific Computation}

\begin{itemize*}
\item Developed and deployed data integration software for laboratory
  automation platforms (HPLC, IR, spectrophotometer).
\item Rapidly implemented ideas into software to support data
  analysis, molecular modeling, analytical chemistry method
  development, platform QA, etc.
\end{itemize*}

\leftandright{Williams College, Dept.\ of Biology}{Williamstown, MA
  \textbullet\, June -- August 2005} \\
\textit{Research Assistant}

\begin{itemize*}
\item Planned and implemented experimental design and data analysis
  for field research project.
\item Participated in field work, data collection, and data entry.
\end{itemize*}

\leftandright{University of Minnesota, Dept.\ of Ecology and
  Evolutionary Biology}{Saint Paul, MN \textbullet\, June -- August
  2004} \\
\textit{Research Intern (NSF supported)} \\
\leftandright{University of Minnesota, Dept.\ of Ecology and
  Evolutionary Biology}{Saint Paul, MN \textbullet\, June -- August
  2003} \\
\textit{Research Intern}

\begin{itemize*}
\item Developed and analyzed numerical results from novel model for ecological
  community assembly model.
\item Participated in field work, data collection, and data entry.
\end{itemize*}

\vspace{1.0em}

\ressection{Relevant coursework}

\begin{itemize*}
\item Foundational computer science, e.g., Data Structures,
  Design/Analysis of Algorithms, and Programming Languages.
\item Bayesian Stats, Biostats, graduate Math Stats I/II, graduate
  Probability/Stochastics, Recent Applications in Probability and Statistics.
\item Seminars in Grapical Models and in Bayesian Nonparameterics.
\item Several courses in Computational Biology, covering use of
  dynamic programming and approximation algorithms to efficiently
  learn structure from data.
\end{itemize*}

\vspace{1.0em}

\ressection{Programming}

\begin{itemize*}
\item Python, R, C/C++, Matlab, Haskell, Scheme, Java.
\item Experience with standard statistics, numerics, and
  visualization packages.
\item Comfortable with software engineering stack (UNIX, git/svn,
  make, etc.), scientific computation stack (MPI, reproducible
  research, \LaTeX, etc.), and web development stack (client-server,
  SQL, HTML, CSS, JavaScript, etc.).
\end{itemize*}

\end{document}
